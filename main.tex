%==========================================================
% DO NOT MODIFY THIS FILE <================================
%==========================================================
%
% This is a LATEX template for the Homework in MATH 352
% Just add your name in the space provided
% and type your solutions to each problem in the space below
%
%==========================================================
% DO NOT CHANGE ANYTHING ELSE <============================
%==========================================================
\documentclass[12pt]{article}
%========================
\def\cdate{{Fall 2020}}
%========================
\usepackage{amsmath,amsfonts,amssymb,txfonts,color}
\textwidth=17truecm
\textheight=20truecm
\oddsidemargin=0pt
\evensidemargin=0pt
%\parindent=0pt
\makeatother
\pagestyle{myheadings}
\markboth{\small\it
Homework for CSE 3044: Update \cdate}
{\small\it 
Homework for CSE 3044: Update \cdate}
\makeatletter
\renewcommand{\@evenhead}{\raisebox{0pt}[\headheight][0pt]{\vbox{\hbox
to \textwidth{\thepage\hfil\strut\leftmark}\hrule}}}
\renewcommand{\@oddhead}{\raisebox{0pt}[\headheight][0pt]{\vbox{\hbox
to \textwidth{\rightmark\hfil\strut\thepage}\hrule}}}
\makeatother

\usepackage{algpseudocode}
\def\tab{\hspace*{10mm}}
\def\halftab{\hspace*{5mm}}
%=========================================================
\begin{document}
\begin{titlepage}
    %\thispagestyle{empty}
    \null
    \vspace{-40mm}
    \hspace*{90truemm}{\hrulefill}\par\vskip-4truemm\par
    \hspace*{90truemm}{\hrulefill}\par\vskip5mm\par
    \hspace*{90truemm}{{\large\sc New Mexico Tech {\rm (\cdate)}}}\vskip4mm\par
    \hspace*{90truemm}{\hrulefill}\par\vskip-4truemm\par
    \hspace*{90truemm}{\hrulefill}
    \par
    \bigskip
    \bigskip
    \par
    \vspace{3cm}
    \centerline{\Large\bf CSE 3044: INSERt}
    \bigskip
    \centerline{\large\bf
    }

    \bigskip
    \centerline{\large\bf
        %================================================================
        % INSERT YOUR NAME ON THE LINE BELOW <=====================
        %================================================================
        Kassidy Maberry
    }
    \bigskip
    \centerline{\it New Mexico Institute of Mining and Technology}
    \centerline{\it Socorro, NM 87801, USA}
    \centerline{\it E-mail:
        %================================================================
        % INSERT YOUR EMAIL ON THE LINE BELOW <=====================
        %================================================================
        kassidy.maberry@student.nmt.edu
    }
    \bigskip
    \centerline{Date:
        %================================================================
        % INSERT THE DATE ON THE LINE BELOW <============================
        %================================================================
        2024/10/07
    }

    \vfill
\end{titlepage}
%=============================================================

\par
\bigskip
{\bf Problem
    %================================================================
    % INSERT PROBLEM NUMBER ON THE LINE BELOW <=====================
    %================================================================
    1
}

\par
\bigskip
{\bf Proof:}
%==================================================================
% INSERT YOUR SOLUTION HERE
%==================================================================
% write your solution in the space below <=========================
% you may add as many lines as needed =============================
%==================================================================
\par
Think of when f(n) = 0...


%==================================================================
\par
%==================================================================

\par
\bigskip
{\bf Problem
    %================================================================
    % INSERT PROBLEM NUMBER ON THE LINE BELOW <=====================
    %================================================================
    2
}

\par
\bigskip
{\bf Proof:}
%==================================================================
% INSERT YOUR SOLUTION HERE
%==================================================================
% write your solution in the space below <=========================
% you may add as many lines as needed =============================
%==================================================================
\par
\begin{tabular}{|c|c|c|}
    \hline
    Line & Cost & Times\\
    \hline
    1 & c & 1\\
    2 & c & n\\
    3 & c & n - 1\\
    4 & c & X\\
    5 & c & Y\\
    6 & c & Y\\
    7 & 3c & n - 1\\
    \hline
\end{tabular}\\
Since this is worst case time complexity we will assume that 
The if statement will always be true upon execution thus 
lines 5 and 6 have the same number of times ran.\\\\
We'll start with the following expression for $T(n):$\\
$T(n) = c + cn + c(n - 1) + cX + 2cY + 3c(n-1)$.\\
$T(n) = cX + 2cY + cn + 4c(n - 1) + c$.\\
First we will need to determine the runtime of X.\\
We know that at minimum line 4 will execute n times to 
check the for loop to execute.\\
X will then execute $Z = (n - 1) + (n - 2) + (n - 3) + ... + 1$.\\
As the required amount of comparisons will decrease by 1 each time.\\
Thus, we can rewrite X as $X = \Sigma^{n - 1}_{i = 1} (n - i)$.\\
We can compute the sum of X as $X = \frac{n(n - 1)}{2}$.\\
Now we just need to compute the sum of Y.\\
Each time, lines 5, 6 are executed we know it will 
execute one less time than X because the final execution 
of X is the terminating condition.\\
$Y = \Sigma^{n - 1}_{i = 1} ((n - i) - 1)$\\
Or $Y = \Sigma^{n - 1}_{i = 1} (n - i) - \Sigma^{n - 1}_{i = 1} 1$.\\
Then $Y = \frac{n(n - 1)}{2} - \Sigma^{n - 1}_{i = 1} 1$.\\
Summing 1 n-1 times gives us n-1 with a final result of.\\
$Y = \frac{n(n - 1)}{2} - n + 1$.\\
$Y = \frac{n^{2} - n}{2} - n + 1$.\\
$Y = \frac{n^{2} - 3}{2} + 1$.\\
Finally, we can substitue back in giving us T(n).\\
$T(n) = c(\frac{n^{2} - n}{2}) + c(n^{2} - 3n) + 2c + cn + 4c(n - 1) + c$
$T(n) = 3c\frac{n^{2} + n}{2} - 5c$.\\


%==================================================================
% This symbol indicates the end of the proof.
%==================================================================
\rightline{$\blacksquare$}

%==================================================================
\par
%==================================================================

\par
\bigskip
{\bf Problem
    %================================================================
    % INSERT PROBLEM NUMBER ON THE LINE BELOW <=====================
    %================================================================
    3a
}
Using our pseudo-code notation, write an algorithm HEAPDELETE(A, i) that
deletes the data in node number i and runs in O(lg n) time for an n-element
heap. Of course, A remains a heap after deletion.
It is crucial that you explain your algorithm clearly using comments. The grader
will assume that the algorithm is incorrect if it is not clear.
(By default, heaps are max-heaps, i.e., descending heaps.)

\par
\bigskip
{\bf Proof:}
%==================================================================
% INSERT YOUR SOLUTION HERE
%==================================================================
% write your solution in the space below <=========================
% you may add as many lines as needed =============================
%==================================================================
\par
\hline
HEAPDELETE(A, i)
\hline
\begin{algorithmic}
    \State $s \gets A.size$
    \State $A[i] \gets A[s]$ \coment{Assign A[i] as the last most element of the heap}
    \State $A.size \gets A.size - 1$ \comment{Resize heap}
    \While{$i > 1$} \comment{We know the root will always be the largest element}
        \State {Heapify(A,i)} \comment{Restore the heap property if needed}
        \state $i \gets \lfloor\frac{i}{2}\rfloor$ \comment{Get next parent}
\end{algorithmic}

%==================================================================
% This symbol indicates the end of the proof.
%==================================================================
\rightline{$\blacksquare$}

%==================================================================
\par
%==================================================================


\par
\bigskip
{\bf Problem
    %================================================================
    % INSERT PROBLEM NUMBER ON THE LINE BELOW <=====================
    %================================================================
    3b
}
Explain (no need to prove) why your algorithm attains the O(lg n) bound.

\par
\bigskip
{\bf Proof:}
%==================================================================
% INSERT YOUR SOLUTION HERE
%==================================================================
% write your solution in the space below <=========================
% you may add as many lines as needed =============================
%==================================================================
\par
\hline
We know heapify has a runtime of $O(lg(n))$ if the heap property 
needs to be restored initially at a given node i. Otherwise, 
we need to check the parent of i. Since the heap property is the 
previous lists at this point is assumed to have held true the 
runtime of heapify is then $O(1)$ and instead relies on the 
amount of nodes we need to run heapify on. We know at most 
the amount of nodes required will be the height which is $lg(n)$.
Thus, $O(lg(n))$ is our runtime in that case. Thus, in both cases 
our runtime is $O(log(n))$.\\

%==================================================================
% This symbol indicates the end of the proof.
%==================================================================
\rightline{$\blacksquare$}

%==================================================================
\par
%==================================================================

\par
\bigskip
{\bf Problem
    %================================================================
    % INSERT PROBLEM NUMBER ON THE LINE BELOW <=====================
    %================================================================
    4
}
Full proof for heapsort
\par
\bigskip
{\bf Proof:}
%==================================================================
% INSERT YOUR SOLUTION HERE
%==================================================================
% write your solution in the space below <=========================
% you may add as many lines as needed =============================
%==================================================================
\par
First let us look at the heap sort algorithm.\\
\hline
Heapsort(A)
\hline
\begin{algorithmic}
    \State $n \gets A.heapsize$
    \State BuildHeap(A)
    \For $i \gets n$ downto 2
        \State Exchange(A, 1, i)
        \State A.heapsize $gets $ A.heapsize - 1
        \State Heapify(A, 1)
\end{algorithmic}

We can construct a $T(n)$ of Heapsort with our given algorithm.
We will represent BuildHeap as $B(n)$ and heapify as $h(n)$.
Assuming each statement has a cost of c.\\

$T(n) = nc + (n-1)h(n)c + 2c(n - 1) b(n)c + c$.\\

Let's begin by deteriming what $h(n)$ is. Starting by looking at the 
heapify algorithm\\
\hline
Heapify(A ,i)
\hline
\begin{algorithmic}
    \State $l \gets left(i)$
    \State $r \gets right(i)$
    \If $l \ge A.heapsize$ and $A[l] > A[i]$
        \state $largest \gets l$
    \If $r \ge A.heapsize$ and $A[r] > A[largest]$
        \state $largest \gets l$
    \If $largest <> i$
        \state swap(A, i, largest)
        \state heapify(A, largest)
\end{algorithmic}

%==================================================================
% This symbol indicates the end of the proof.
%==================================================================
\rightline{$\blacksquare$}

%==================================================================
\par
%==================================================================

\end{document}